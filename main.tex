\documentclass{article}
\usepackage[utf8]{inputenc}

\usepackage{vmargin}

\setpapersize{A4}
\setmargins{2.54cm}       
{2.54cm}                        
{15.5cm}                      
{21.42cm}                    
{5pt}                           
{1cm}                           
{5pt}                            
{2cm} 

\title{Interrupciones}
\date{Julio 2020}

\usepackage{natbib}
\usepackage{graphicx}

\begin{document}


\begin{titlepage}
\centering
{\bfseries\LARGE Universidad de Atioquia\par}
\vspace{1cm}
{\scshape\Large Facultad de Ingenier\'ia Electr\'onica \par}
\vspace{3cm}
{\scshape\Huge Interrupciones \par}
\vspace{3cm}
{\itshape\Large Proyecto de Investigaci\'on \par}
\vfill
{\Large Autor: \par}
{\Large Daniel Felipe Y\'epez Taimal \par}
\vfill
{\Large Julio 2020 \par}
\end{titlepage}

\maketitle

Actualmente las interrupciones han sido de mucha ayuda en la electrónica, puesto que, han ayudado en la optimización de la gestión de los dispositivos de entrada y salida. A pesar de que las interrupciones son un concepto que siempre ha existido en los procesadores y microprocesadores, estas han pasado por varios metodos para ser procesadas e identificadas, por ejemplo, para gestionar los dispositivos I/O, se utilizaba un método simple pero ineficiente llamado polled,\cite{Mar} donde el procesador recibía una solicitud de interrupción, pero no especifican en qué lugar, por lo que el procesador debía revisar cada uno para verificar quien hizo la solicitud de interrupción, esto a pesar de que solucionaba el problema significaba gastar gran cantidad de tiempo que podría ser implementado en otras tareas. Gracias a esto las interrupciones se han convertido en la mejor forma de gestionar estos procesos.\\

Las interrupciones se podrían definir como un \cite{Dar} "artificio hardware/software", específicamente es una señal que indica que se debe interrumpir el curso de ejecución actual y ejecutar un código especifico, así pues, las interrupciones tienen dos niveles el Hardware y el Software. \cite{Jho1} En el primer nivel un dispositivo genera la solicitud de interrupción, entonces el microprocesador, tiene que terminar la instrucción que se encuentra realizando, y después de concluida esta instrucción se realiza un salto a una subrutina donde se efectúa el tratamiento de la interrupción; por la parte del Software, justo cuando se termina el proceso antes de la interrupción, se guarda esa dirección de memoria, se procesa la interrupción y se reanuda con la instrucción del programa principal.\\

En las implementaciones de la cotidianidad, los procesos no suelen ser tan simples, por lo que presentan situaciones donde al microprocesador se \cite{Jho2} le solicita una interrupción mientras otra interrupción esta en espera; para estos casos se utiliza un nivel de jerarquía de las interrupciones que llegan al microprocesador, a nivel del Hardware encontramos a las Enmascarables y las no enmascarables. Las interrupciones de tipo Hardware no son programadas y pueden ocurrir en cualquier momento, dentro de ellas las de mayor prioridad son las interrupciones no enmascarables, aun así el procesador puede evitar atenderlas (no enmascarables), no atenderlas o ignorarlas (enmascarables); por parte del Software las interrupciones son programadas y normalmente son usadas para hacer uso de los dispositivos  I/O, estas interrupciones son programadas por el usuario y tienen una prioridad muy alta (por encima de las interrupciones de Hardware).\\

La detección de las interrupciones se puede hacer por medio de hardware y software. En el hardware podemos encontrar varios métodos, como las \cite{Ser} "líneas INT/IRQ", donde un procesador tiene varias entradas INT y a cada una se le conecta un dispositivo I/O y cuando se hace la solicitud de interrupción el controlador debe seguir la ruta generada y ejecutar la interrupción; el mecanismo INT/INTA, el cual consistía en que el procesador tiene solamente una entrada para el pedido de una interrupción, este mecanismo establecía un nivel de jerarquía implícito, puesto que el controlador que este más cerca seria atendido primero, este mecanismo fue un fracaso comercial puesto que exigía que todos los controladores debía tener una misma arquitectura, después se genero el concepto de controlador de interrupciones, este concepto al ser más universal permitía que controladores de diferentes arquitecturas usaran el mecanismo INT/INTA, gracias a un controlador de interrupciones.\\

\cite{Dan} "Las interrupciones añaden cierta complejidad al diseño del Hardware", así pues ha salido al mercado muchos controladores especializados en interrupciones específicas y con una arquitectura específica, como lo son por ejemplo los PC poseen un hardware orientado por completo a la multitarea, los teclados mecánicos y se ha generado una carrera en los circuitos integrados para ver cual puede aceptar la mayor cantidad de interrupciones  posibles, así pues el hardware se ha convertido en parte fundamental para el manejo de sistemas con interrupciones.\\

La detección de Software es creada para que el programador pueda acceder a servicios propios del sistema operativo, un ejemplo es la \cite{Ser} "rutina de exploración de teclado que denominaremos TRAP 0, donde después de comprobar si alguna tecla fue pulsada devuelve un registro donde se especifica la ruta, así pues, cada vez que algún programa desee acceder a información proporcionada por el teclado simplemente utilizara la rutina TRAP 0", entonces el lenguaje de programación al igual que el hardware se ve vitalmente involucrado en el manejo de las interrupciones, puesto que al  estar directamente involucrado en la manera de administrar los recursos del procesador se requiere que dichos recursos se usen de la mejor manera; el mejor ejemplo de esta situación es el \cite{Dan}microprocesador 8086 el cual es compatible con PC multitarea, pero este se limita a chequear continuamente las variables del disco hasta que estas cambien, lo cual significa un claro desaprovechamiento  de las posibilidades de gestión que nos brindan las interrupciones.\\

En manera de resumen, las interrupciones son un recurso de vital importancia, nos permiten recibir y devolver datos, detectar incongruencias en los procesos, entre otras cosas. Por eso, sé necesita comprender el funcionamiento de estas desde el más alto hasta el más bajo nivel, porque aunque son un recurso importante, un mal manejo de estas puede desembocar en un sistema ineficiente y en gastos innecesario de recursos, tanto de software como de hardware o en caso de ignorar por completo el nivel de jerarquía, tener un funcionamiento no deseado en nuestro sistema.\\



\newpage

\bibliographystyle{plain}
\begin{thebibliography}{X}
\bibitem{Dar} \textsc{Dario Alejandro Alpern D.A.A},
\textit{El microprocesador 8085}, Buenos Aires, Argentina.
\bibitem{Dan} \textsc{Daniel Ulises Campos Delgado D.C},
\textit{ EL CONTROLADOR DE INTERRUPCIONES 8259}, San Luis Potosí, Mexico.
\bibitem{Ser} \textsc{Sergio de Cola S.C},\textit{ Interrupciones}, Montevideo, Uruguay.
\bibitem{Mar} \textsc{Margaret Rouse M.R},\textit{polled interrupt}, Washington, USA.
\bibitem{Jho1} \textsc{john alexander sanabria ordonez J.S},\textit{OS - C01 - S05 - Gestión de interrupciones}, Cali,Colombia.
\bibitem{Jho2} \textsc{john alexander sanabria ordonez J.S},\textit{OS - C01 - S06 - Gestión de interrupciones}, Cali,Colombia.

\end{thebibliography}
\end{document}
